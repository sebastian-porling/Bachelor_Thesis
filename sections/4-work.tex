\section{The conducting of usability tests, and designing and implementing the MCDM }


\subsection{Designing the decision model}

\subsection{Conducting comparative usability tests}
Before each test, the subjects answered some background questions. These were used both to gather more information for the statistical review of the test results, and to make the subjects more open and ready to talk.
\newline
\renewcommand{\arraystretch}{1.5}
\begin{table}[ht!]
    \centering
    \begin{tabularx}{0.9\textwidth} { 
        | >{\raggedright\arraybackslash}X 
        | >{\centering\arraybackslash}X 
        | >{\raggedleft\arraybackslash}X | }
        \hline
        What is your main occupation? \\
        How much time do you spend on your smartphone on an average day? \\
        How much of that time do you spend on applications versus websites? \\
        What kind of mobile applications do you mostly use? Work, chat, social media, finances,  shopping etc.?\\
        \hline
    \end{tabularx}
    \caption{\label{tab:background-questions} Background questions for the usability tests.}
\end{table}

For the user tests, the Nielsen Norman Group’s article on task scenarios for usability tests was used. The scenarios were designed to be relatively realistic, actionable and with an appropriate level of vagueness for the test subjects to perform the tasks. The scenarios in table \ref{tab:usability-test-tasks} were included in the first tests.

\begin{table}[ht!]
    \centering
    \begin{tabularx}{0.9\textwidth} { 
        | >{\raggedright\arraybackslash}X 
        | >{\centering\arraybackslash}X 
        | >{\raggedleft\arraybackslash}X | }
        \hline
        1. Find the recipe “Avocado vegan fudge brownie” \\
        2. Use the app to find recipes with chicken which take less than 15 minutes to cook \\
        3. You are making dinner. Use the app to make a shopping list for; a dish with fish; a dish with bacon and a dessert \\
        4. One can add their dietary preferences/allergies in the app. Try adding that you eat gluten-free and vegan, and find recipes for chilli. Browse around and find a recipe you find interesting. \\
        5. It’s possible to save recipes through a so-called “Yum’s” list. Create a “Yum’s” list called Pastries and add a few recipes to the list \\
        6. Add “milk” and “eggs” to the shopping list \\
        7. You’re in the store and have now picked up eggs, update your shopping list. \\
        \hline
    \end{tabularx}
    \caption{\label{tab:usability-test-tasks} The first list of tasks.}
\end{table}

The tests were recorded with a camera, showing the face and recording the voice of the test subject. This allowed for richer, more nuanced qualitative data to be gathered from the usability tests, and also allows for tests to be revisited and analyzed a posteriori. This was useful since there were only two people conducting the tests, and neither has extensive experience in usability testing. \\
Pilot tests were performed first to assess how well the thought out scenarios work. The pilot tests were timed to evaluate how time-consuming the different parts of the test were. These pilot tests averaged at 30-40 minutes. Some scenarios were removed and edited to minimize time consumption, without affecting the quality of the results. The order of the tasks was also changed to better mimic how the application would be used. Scenario number 6 was put first for this reason.  Scenario number 1 was removed because scenario 2 and 4 filled the same function.

\begin{figure}[ht!]
	\centering 
    \includegraphics[width=0.8\textwidth]{img/placeholder.png}
	\hfill
	\caption{\textit{ Graphics/diagrams for SUS and UEQ }}
\end{figure}

The SUS, UEQ and consent forms were intitally printed on paper to ease the process of filling them in. After the pilot tests, this was changed to have the subjects fill in the form on the computer instead. This saved both time and the amount of paper used for the tests. The data from the SUS and UEQ was stored digitally to be used for statistics. The tests were saved separately depending on the device used and type of application tested. This would ease the statistical analysis between the platforms. The aspect of Novelty from the UEQ was left out of this test, as it was not within the scope of the research question.

\begin{figure}[ht!]
	\centering 
    \includegraphics[width=0.8\textwidth]{img/placeholder.png}
	\hfill
	\caption{\textit{ UML of how the tests were conducted }}
\end{figure}

The test subjects were recruited through a form shared on social media, and posted on bulletin boards at the campus of KTH. The test subjects were compensated with a goodie-bag and free coffee. 

\subsection{Evaluating the decision model}
