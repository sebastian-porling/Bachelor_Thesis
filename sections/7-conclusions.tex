\section{Conclusions}


In this chapter, the final conclusions on the projects findings are presented. Finally, in the section Future Work, research which could be conducted in the future on the topic of this project is discussed. 

\subsection{Conclusion}

\textit{Do users notice a difference between progressive web applications and native apps, and does the difference affect the user experience?}

According to our findings yes, to a certain extent. The biggest difference between PWAs and native applications was at the time of writing the limitations on functions available, but even when functionality is not taken into consideration we would still argue that PWA was shown to be worse from an end-users perspective. We say this because the difference between the native application and PWA was significantly noticeable when it came to response time, at least on the older device tested. It was possible we could have gotten a different result if we had tested with a different application. When trying to find an application to perform the tests on, another product called Pinterest was considered. However, the PWA for Pinterest did not work on the iOS device available so it was not selected for this test. The PWA on Pinterest was more similar to the native application and would otherwise have been a good choice for the tests. Our conclusion from the usability tests is that when the device running the application has outdated hardware the PWA would not perform as well as a native application. However, it was also possible that the PWA for Yummly was simply not as well-implemented as its native counterpart.

Something to take into account is that from the survey we noticed that half or more of users will probably try to find the application through the phones app store first. If this was a trend among the population in general this could mean that users would need to do more searching in order to find the PWA than a native application. There also seems to be a trend that iPhone users primarily find information about applications through app store. However, this was not a big enough survey to convey a definite conclusion on the matter. 

\textit{Can a model accurately decide whether a PWA, a React Native application or a native application is most favourable?}

Yes and no. After presenting the model to Slagkryssaren AB the consultants expressed the opinion that the model seemed to give accurate results based on the input given. The recommendation from the model in itself was not a definite truth, but the recommendation in combination with process of filling in the form gives a good ground for discussion. This could be helpful to challenge a fixed idea of what application to implement, either within the development team or when discussing with a customer.

The subject of choosing a development tool is subjective in its nature, it was therefore not unexpected that the model would be used more as a tool for discussion and prioritization than a definite decision-making tool.

\subsection{Future Work}

A usability test including PWAs, native applications and hybrid applications such as React Native could further improve the reliability of a decision model, as the UX criteria for React Native in this model was based entirely on opinion from developers at Slagkryssaren AB.

The score for the criteria Maintenance was also based on opinions from developers. An improvement would, therefore, include basing this number on some kind of statistics, analyzing how much more maintenance a PWA needs compared to other applications. For the Budget criterion to be more precise, research on how the revenue is affected by where an application is downloaded and which payment methods the end-users can utilize is necessary.

A larger study on application exploration could also be enlightening in the choice of application development technique. It would be useful to compare how different demographics find new applications, to find out the true impact of an application being present or absent on the most popular app stores.
