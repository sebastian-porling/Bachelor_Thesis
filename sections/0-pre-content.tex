\newpage

%%%%%%%%%%%%%%%%%%%%%%%%%%%%%%%%%%%%
%%														 The English abstract						            		%%
%%%%%%%%%%%%%%%%%%%%%%%%%%%%%%%%%%%%
\section*{Abstract}
%%%%%%%%%%%%%%%%%%%%%%%%%%%%%%%%%%%%
There are many possible tools to develop mobile applications with. Choosing a development tool is done by considering many different factors, and the choice is currently done, in many cases, arbitrarily. For this project, a decision model is designed to ease the process of choosing a development tool.

A survey was conducted to examine how people using different smartphone platforms discover and download applications. 94 responses were collected, showing that approx. 50\% of Android-users found mobile applications by using search engines or browsers. This number was approx. 30\% for iOS-users.

A usability test was conducted to discover the differences in user experience between Progressive web applications and native applications. 18 usability tests were conducted comparing the same product developed as a Progressive Web Application and a native application. Both Android and iOS devices were included in the tests.

The results indicated that end-users notice when an application is not natively developed. The effect on the user experience is combined with other technical differences and applied to the decision model. This model was designed to predict if a native application, a Progressive Web Application or a React Native application is the most favourable to develop for a specific scenario.

The final model could, according to consultants at the stakeholder Slagkryssaren AB, with good accuracy predict when the different development tools should be used. The model could be used as a discussion tool in the first stages of the development process of an application. 

\subsection*{Keywords}
Progressive Web Application, User experience, Native application, Hybrid application, Usability testing, React Native, Decision model, Weighted Sum Model, Multi-Criteria Decision-Method






\newpage
%%%%%%%%%%%%%%%%%%%%%%%%%%%%%%%%%%%%
%%														 The Swedish abstract								         %%
%%%%%%%%%%%%%%%%%%%%%%%%%%%%%%%%%%%%
\section*{Sammanfattning}
%%%%%%%%%%%%%%%%%%%%%%%%%%%%%%%%%%%%
Det finns många möjliga verktyg för att utveckla mobila applikationer. Valet av utvecklingsverktyg görs genom att överväga många olika faktorer, och görs idag i många fall högst godtyckligt. För det här projektet designades en beslutsmodell som förenklar processen av att välja ett utecklingsverktyg.

En undersökning gjordes för att undersöka hur användare av olika smartphone-plattformar upptäcker och laddar ner applikationer. 94 svar samlades, svaren visade att ungefär 50\% av Android-användare hittade mobila applikationer genom internetsökningar eller webbläsare. Denna siffran var ungefär 30\% för iOS-användare.

Ett användarbarhetstest utfördes för att finna skillnader i användarupplevelse mellan progressiva webbapplikationer och native-applikationer. Både Android- och iOS-enheter testades.

Resultatet tydde på att slutanvändare la märke till när en applikation inte utvecklades som en native-applikation.\\ Effekten på användarvänligheten, kombinerat med tekniska skillnader mellan verktygen, tillämpades på beslutsmodellen. Modellen designades för att förutse om en native-applikation, en progressiva webbapplikation eller en React Native-applikation är mest fördelaktig att utveckla i ett specifikt scenario.

Den slutgiltiga modellen kunde, enligt konsulter på uppdragsgivaren Slagkryssaren AB, med god precision avgöra när de olika utvecklingsverktygen bör nyttjas. Modellens användning blev som ett diskussionsverktyg i de första stadierna av processen med att välja utvecklingsvektyg.


\subsection*{Nyckelord}

Progressiv Webapplikation, Användarupplevelse, Nativeapplikation,\\ Hybridapplikation, Användbarhetstestning, React Native, Beslutsmodell, Viktad Summa Model, Multikriteriebeslutsmetod


\newpage
\section*{Acknowledgements}
We would like to thank Slagkryssaren AB for allowing us to conduct this study at their facilities and sponsoring us with goodie-bags. We would especially like to thank Oskar, our supervisor at Slagkryssaren AB, for providing understanding of the business perspective of this project. We would also like to thank Fredrik Kilander, our academic supervisor, and Anders Sjögren, our examiner, for providing guidance through every predicament on the way. Finally, we would like to thank all the participation in the market survey and the usability tests.






\newpage
\thispagestyle{empty}

~\\
\vfill
{ \setstretch{1.1}
	\subsection*{Authors}
	Celine Mileikowsky <celinemi@kth.se> and Sebastian Porling <porling@kth.se>\\
	Electrical Engineering and Computer Science\\
	KTH Royal Institute of Technology
	
	\subsection*{Place for Project}
	Stockholm, Sweden\\
	Slagkryssaren AB
	
	\subsection*{Examiner}
	Anders Sjögren <as@kth.se> \\
	Kista, Sweden\\
	KTH Royal Institute of Technology
	
	\subsection*{Supervisor }
	Fredrik Kilander <fki@kth.se>\\
	Kista, Sweden\\
	KTH Royal Institute of Technology
	~
	
}


