\section{Introduction}

This chapter gives a short introduction and background of the area. It provides what problem will be solved and what the purposes and goals are for the study that will be conducted. It introduces the research methodologies that will be used, along with the stakeholders and this study's delimitations. Finally, it goes through the structure of the report as a whole.

\subsection{Background}

The market for mobile applications is constantly evolving and changing, with new research being made in both performance and usability. With a growth in the market for mobile internet devices, the market has also diversified, leading to many different devices for mobile internet access being available. For a mobile application to penetrate the market properly, it needs to be available across all popular mobile device platforms. 

Progressive web applications provide this important cross-platform compatibility \cite{PWAGoogle} without the need to develop several applications. 
Progressive web applications (PWA) are web applications which are engaging since they can be added to the home screen, as well as fast and reliable since, just like native applications, they can be accessed without an internet connection \cite{PWAGoogle}. The benefit of developing a PWA instead of a native application lies in its reachability and cost-efficiency compared to developing separate native applications for several platforms. Native applications refers to applications written for the top two mobile platforms: iOS, using Swift or Object-C; and Android, using Kotlin or Java. 

Another solution to the platform issue is using a hybrid application, such as React Native. React Native is a mobile application framework which combines React, a JavaScript library, and native platform capabilities to develop applications for the most common platforms.

With a PWA or hybrid application, only one application needs to be developed, and that application can be run on a plethora of platforms. However, this cross-platform compatibility comes with some compromises in functions. A PWA on an iPhone can for example not access push-notifications or advanced hardware, and on all platforms, the graphical design might differ from what is the norm for the platform.

\subsection{Problem}
A problem that has arisen with the diversification of the market is compatibility across the different mobile device platforms. Applications that work on one platform might not be functional on another, and companies are often forced to launch several different applications for different devices. 
A PWA solves this problem, as a developer only has to develop one application, and that application can be run on all platforms. There are, however, some performance issues with a PWA. 
To find out how much compromises in performance of a PWA the end-user notice, user experience was tested.

The user experience effects of choosing the different development tools was combined with the technical advantages and drawbacks of the different application development tools to determine which type of application was beneficial to develop in a specific case.

This study investigated the difference of user experience in PWAs and native applications to answer the following questions:

\begin{itemize}
    \item Do users notice a difference between progressive web applications and native applications, and does the difference affect the user experience?
\end{itemize}

With these findings, combined with other factors, a model was built to answer the question:

\begin{itemize}
    \item Can a model accurately decide whether a PWA, a React Native application or a native application is most favourable?
\end{itemize}

\subsection{Purpose}

Several studies have been made on mobile applications and web applications, but few have taken PWA into account.
The purpose of this study was to define how using a PWA affects the user experience, and weigh that together with other factors into a decision model. The model was to be used as a tool to determine if a PWA, a React Native application or a native application should be developed.

\subsection{Goal}

The goal was to develop a model for software developers that eases the process of choosing between different application development options. The model should use the customers’ and end-users’ preferences and demands as a basis and produce the most suitable option for a mobile application development tool.

\subsubsection{Benefits, Ethics and Sustainability}

The impact of this study was analyzed using the four pillars of sustainability \cite{FourSustainabilityPillars2017}. The UN Sustainable Development Goals  \cite{UNSustainabilityGoals} were reviewed for the planning of this study.

\paragraph{Human sustainability} 

The data that was collected from the usability tests and surveys was to be handled in a way that it could not be accessed or used by any other parties than the authors of this study. The data was to be presented in a way that anonymizes the test subjects.

\paragraph{Social sustainability}

Products being available on all mobile platforms would improve the mobile connectivity aspect of the UN Sustainable Development number 9 Industry, innovation and infrastructure \cite{UNSustainabilityGoals}. This effect, however, was not massive as the technology for these development techniques already exists. 

\paragraph{Economic sustainability}

PWAs are overall easier, cheaper and faster to develop and launch than making two native applications, according to IT-consultants at the company Slagkryssaren AB. Making a model that helps the developers to choose a PWA at appropriate cases was therefore advantageous for the customer, and possibly the end-user.

\paragraph{Environmental sustainability}

This study did not have any foreseeable effects on environmental sustainability. The application development does not change the environmental impact of the application in any way noteworthy.

\subsection{Methodology}

The research questions for this study were\textit{ "Do users notice a difference between progressive web applications and native applications and does the difference affect the user experience?"}
and \textit{"Can a model accurately decide whether a PWA, a React Native application or a native application is most favourable?".}

These questions could be divided into two research sections: User experience and Designing a model.

\subsubsection{User experience}

To understand the user experience differences between PWAs and native applications, the differences between the development methods had to be investigated, along with the experienced differences when using the applications. 
To get this understanding, end-users tested applications made with both tools and provide feedback. The opinions of experts within the area combined with research and literature previously done on similar subjects had to be investigated.  

\subsubsection{Designing a model}

To build a decision model, which criteria are important to the decision had to be established. 
When these criteria were established, how well the different development tools perform on said criteria had to be determined. A method for choosing a development tool had to be decided on and implemented. 

The model implemented with the chosen method was to be evaluated and improved upon until it produced accurate outcomes.

\subsection{Stakeholders}

The model created was designed for the company Slagkryssaren AB.
Slagkryssaren AB is a tech agency that specializes in mobile platforms, back-end programming, cloud computing, artificial neural network technologies and software architecture. With this model, they would be able to help their clients choose an appropriate product that fits their needs more easily. Slagkryssaren AB supplied with their expertise and experience when it comes to choosing what criteria are important, how they usually conduct a decision for choosing between the different technologies and the business perspective of the decision.

\subsection{Delimitations}

The most restrictive resource for this study was time. Only PWA, React Native and native developing were included in the final results. There are at the time of writing many more hybrid solutions for making mobile applications, and there was not time to compare all of them. 

The smartphones that were looked into were running the two most popular operating systems in Sweden \cite{MarketShareSweden2020}, Android and iOS devices. The device was not the test subjects’ own device, so there was a risk of them having to deal with a different layout than they are used to. 

The usability tests was limited by demographic and number of participants, as each test required a lot of time for planning, execution and analysis. The tests were limited by being performed in Stockholm only. The tests were not performed online, so our participants had to be present physically during office hours, which limited the number of participants with day-time occupations. 

\subsection{Outline}

\begin{itemize}
    \item Chapter 2 addresses the background of the different technologies, usability testing and decision-making methods.
    \item Chapter 3 explains in detail how the study will be carried out. This includes how to gather relevant information, how to test user experience and how to design and implement a decision-method.
    \item Chapter 4 addresses the different steps in implementing the decision-method, show what was found in the literature and case study, and show the time-line of the user experience tests.
    \item Chapter 5 presents the result and explain the reasoning for choosing the different criteria.
    \item Chapter 6 presents the conclusion and discussion of this study. This shows what could have been done differently and what future work could be performed.
\end{itemize}
